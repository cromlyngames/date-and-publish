%%%%%%%%%%%%%%%%%%%%%%%%%%%%%%%%%%%%%%%%%
% based on Twenty Seconds Resume/CV
% LaTeX Template
% Version 1.1 (8/1/17)
%
% This template has been downloaded from:
% http://www.LaTeXTemplates.com
%
% Original template author:
% Carmine Spagnuolo (cspagnuolo@unisa.it) with major modifications by 
% Vel (vel@LaTeXTemplates.com)
%
% License:% The MIT License (see included LICENSE file)
%
%%%%%%%%%%%%%%%%%%%%%%%%%%%%%%%%%%%%%%%%%

%----------------------------------------------------------------------------------------
%	PACKAGES AND OTHER DOCUMENT CONFIGURATIONS
%----------------------------------------------------------------------------------------

\documentclass[letterpaper]{twentysecondcv} % a4paper for A4

%----------------------------------------------------------------------------------------
%	SIDE BAR
%----------------------------------------------------------------------------------------

% If you don't need one or more of the below, just remove the content leaving the command, e.g. \cvnumberphone{}

\profilepic{profilepic.png} % Profile picture

\cvname{< \_\_\_\_\_\_\_\_\_\_\_> } % Your name
\cvjobtitle{A bold millennial who is: } % Job title/career

\cvdate{sporty, boho, grunge, preppy, fem, fashionista, \_\_\_\_ } % Style
\cvaddress{warm, huggy, calm, curious, carefree, engaged \_\_\_\_} % Body type
\cvnumberphone{rude, impetuous, panicky, avoidant, brooding, withdrawn, \_\_\_\_} % flaws
\cvsite{NY rough, NY rich, other city, small town, `burbs, foreign, \_\_\_\_} % From
\cvmail{tech wizard, digital native, old-schooler, fumbling, \_\_\_\_} % Tech comfortable


%----------------------------------------------------------------------------------------

\begin{document}

%----------------------------------------------------------------------------------------
%	 SIDEBAR CONTINUES
%----------------------------------------------------------------------------------------

\aboutme{\emph{Underline one/two   } activist, assistant, copy-editor, designer, executive, graphics, influencer, jeweller, lawyer, photographer,  psychologist, sexpert, stylist, trend setter, writer  } % To have no About Me section, just remove all the text and leave \aboutme{}

% Skill bar section, each skill must have a value between 0 an 6 (float)
%\skills{{pursuer of rabbits/5.8},{good manners/4},{outgoing/4.3},{polite/4},{Java/0.01}}
\skills{{Career/0.01}}

%------------------------------------------------

% Skill text section, each skill must have a value between 0 an 6
%\skillstext{{lovely/4},{narcissistic/3}}
\skillstext{}

%----------------------------------------------------------------------------------------

\makeprofile % Print the sidebar

%----------------------------------------------------------------------------------------
%	 MAIN BLOCK
%----------------------------------------------------------------------------------------

\section{Moves}

If you do something in story that triggers a move: \emph{Roll 2d6 + your score + any bonuses.} 
Generally a total below 7 means a frustration, 7-9 is a complicated success and 10+ is a great success.
Start with +1 in three moves

\section{Start of session}
At the start of each session spend 1-2 career to keep things ticking over. If you cannot, or will not, tell the GM.
%You can earn 3 career on a gig, and odd amounts elsewhere. A gig is a big idea/push or initiative. If you want to start an appropriate gig, talk to the GM
%A new character starts with 2 career. 

\section{Preparatory Moves}

\begin{twenty} % Environment for a list with descriptions
	\twentyitem{[][][]}{Draw on the past}{Mine backstory}{}
	\twentyitem{[][][]}{Lean in}{Help Friends}{}
	\twentyitem{[][][]}{Read the vibes}{Get infomation}{}
	%\twentyitem{<dates>}{<title>}{<location>}{<description>}
\end{twenty}

\section{Drama Moves}

\begin{twenty} 
	\twentyitem{[][][]}{Act out of type}{Once per session}{}
	\twentyitem{[][][]}{Call their shit out}{Bonuses available}{}
	\twentyitem{[][][]}{Navigate Romance}{Bonuses avaialbe}{}
	\twentyitem{[][][]}{Produce the goods}{Career challenges}{}
	\twentyitem{[][][]}{Turn up the charm}{That'd be nice}{}
\end{twenty}

\section{Recovery Moves}

\begin{twenty} 
	\twentyitem{[][][]}{Ruminate on life}{Clear strains}{}
\end{twenty}

\section{Established Connections}

\begin{twenty} 
	\twentyitem{[][][] }{ \_\_\_\_\_\_\_\_\_\_\_\_\_\_\_\_\_\_\_\_\_\_\_\_\_\_\_\_\_\_\_\_}{Strained? [] }{}
	\twentyitem{[][][] }{ \_\_\_\_\_\_\_\_\_\_\_\_\_\_\_\_\_\_\_\_\_\_\_\_\_\_\_\_\_\_\_\_}{Strained? [] }{}
	\twentyitem{[][][] }{ \_\_\_\_\_\_\_\_\_\_\_\_\_\_\_\_\_\_\_\_\_\_\_\_\_\_\_\_\_\_\_\_}{Strained? [] }{}
	\twentyitem{[][][] }{ \_\_\_\_\_\_\_\_\_\_\_\_\_\_\_\_\_\_\_\_\_\_\_\_\_\_\_\_\_\_\_\_}{Strained? [] }{}
	\twentyitem{[][][] }{ \_\_\_\_\_\_\_\_\_\_\_\_\_\_\_\_\_\_\_\_\_\_\_\_\_\_\_\_\_\_\_\_}{Strained? [] }{}
	\twentyitem{[][][] }{ \_\_\_\_\_\_\_\_\_\_\_\_\_\_\_\_\_\_\_\_\_\_\_\_\_\_\_\_\_\_\_\_}{Strained? [] }{}

\end{twenty}



\section{Established Backstory Resources}
{. . . . . . . . . . . . . . . . . . . . . . . . . . . . . . . . . . . . . . . . . . . . . . . . . . . . . . . . . . . . . . . . . . . . . . . . . . . . . . . . . . . . . . . . . . . . . . . . . . . . . . . . . . . . . . . . . . . . . . . . . . . . . . . . . . . . . . . . . . . . . . . . . . . . . . . . . . . . . . . . . . . . . . . . . . . . . . . . . . . . . . . . . . . . . . . . . . . . . . . . . . . . . . . . . . . . . . .}
{. . . . . . . . . . . . . . . . . . . . . . . . . . . . . . . . . . . . . . . . . . . . . . . . . . . . . . . . . . . . . . . . . . . . . . . . . . . . . . . . . . . . . . . . . . . . . . . . . . . . . . . . . . . . . . . . . . . . . . . . . . . . . . . . . . . . . . . . . . . . . . . . . . . . . . . . . . . . . . . . . . . . . . . . . . . . . . . . . . . . . . . . . . . . . . . . . . . . . . . . . . . . . . . . . . . . . . .}






%----------------------------------------------------------------------------------------
%	 EXPERIENCE
%----------------------------------------------------------------------------------------
%
%\section{Experience}
%
%\begin{twenty} % Environment for a list with descriptions
%	\twentyitem{1900}{Alice in Wonderland-The Circra (1900's) Silent Film.}{Film}{The first Alice on film was over a hundred years ago.}
%	\twentyitem{1933}{Alice in Wonderland 1933 version.}{Film}{This film stars Ethel griffies and Charlotte Henry. It was a box office flop when it was released.}
%	\twentyitem{1951}{Disney Film.}{Film}{Walt Disney brings Lewis Carroll's fantasy story to life in this well done animated classic. Even though many elements from the book were dropped, such as the duchess with the baby pig and mock turtle, this version is without a doubt the most famous Alice adaption made.}
%	%\twentyitem{<dates>}{<title>}{<location>}{<description>}
%\end{twenty}
%
%%----------------------------------------------------------------------------------------
%%	 OTHER INFORMATION
%%----------------------------------------------------------------------------------------
%
%\section{Other information}
%
%\subsection{Review}
%
%Alice approaches Wonderland as an anthropologist, but maintains a strong sense of noblesse oblige that comes with her class status. She has confidence in her social position, education, and the Victorian virtue of good manners. Alice has a feeling of entitlement, particularly when comparing herself to Mabel, whom she declares has a ``poky little house," and no toys. Additionally, she flaunts her limited information base with anyone who will listen and becomes increasingly obsessed with the importance of good manners as she deals with the rude creatures of Wonderland. Alice maintains a superior attitude and behaves with solicitous indulgence toward those she believes are less privileged.

%----------------------------------------------------------------------------------------
%	 SECOND PAGE EXAMPLE
%----------------------------------------------------------------------------------------

\newpage % Start a new page

%\makeprofile % Print the sidebar

\section{Other information}

\subsection{Review}

Alice approaches Wonderland as an anthropologist, but maintains a strong sense of noblesse oblige that comes with her class status. She has confidence in her social position, education, and the Victorian virtue of good manners. Alice has a feeling of entitlement, particularly when comparing herself to Mabel, whom she declares has a ``poky little house," and no toys. Additionally, she flaunts her limited information base with anyone who will listen and becomes increasingly obsessed with the importance of good manners as she deals with the rude creatures of Wonderland. Alice maintains a superior attitude and behaves with solicitous indulgence toward those she believes are less privileged.

\section{Other information}

\subsection{Review}

Alice approaches Wonderland as an anthropologist, but maintains a strong sense of noblesse oblige that comes with her class status. She has confidence in her social position, education, and the Victorian virtue of good manners. Alice has a feeling of entitlement, particularly when comparing herself to Mabel, whom she declares has a ``poky little house," and no toys. Additionally, she flaunts her limited information base with anyone who will listen and becomes increasingly obsessed with the importance of good manners as she deals with the rude creatures of Wonderland. Alice maintains a superior attitude and behaves with solicitous indulgence toward those she believes are less privileged.

%----------------------------------------------------------------------------------------

\end{document} 
